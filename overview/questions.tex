\question[3]\label{q:trustcomp}
% examgen: trustcomp:E:C:A
Describe the requirements for a process to be able to assess the integrity of 
itself and its execution environment.

\begin{solution}
  If the process can trust its environment (i.e.\ the operating system), then 
  it can rely on the environment to assess its own integrity.
  Thus the process relies on the integrity of the operating system.
  The oerating system in turn relies on the integrity of the hardware and must 
  rely on the hardware to assess its own integrity.
  Hence the process needs hardware that will not allow a modified version of 
  the operating system to run.
\end{solution}


\question[4]\label{q:trustcomp}
% examgen: trustcomp:E:C:A
Give an example of a DRM system, the idea behind it and why it works or not.

\begin{solution}
  Hardware dongles:
  You have a hardware dongle attached to the computer, the software can then 
  communicate with the dongle.
  The idea is that the software can be copied easily, but the dongle cannot.
  Thus the software can only run in as many instances as there are hardware 
  dongles.

  The hardware dongle can be simulated by other software in many cases.
  For the software to be able to tell the dongle and the simulated dongle 
  apart, it must be able to trust the operating system --- thus it needs to 
  verify the integrity of the operating system, which in turn requires special 
  hardware.
  The alternative approach is that the dongle is more sophisticated, e.g.\ that 
  it uses unforgeable digital signatures as output.
  In this case we instead modify the software itself, so that it simply skips 
  the checks with the dongle (e.g.\ the signature verification always returns 
  true).
\end{solution}


\question[2]\label{q:trustcomp}
% examgen: trustcomp:E:C
What is remote attestation?

\begin{solution}
  Remote attestation is the problem of remotely verify the integrity of 
  a running system.
  Basically to ascertain if a black box runs a given software.
\end{solution}


\question\label{q:trustcomp}
% examgen: trustcomp:E:C:A
\begin{parts}
  \part[2] What is sealed storage?
  \begin{solution}
    This binds the data to the platform, which means that the data is encrypted 
    using the hardware and the key depends on the hardware configuration.
    This means that no one can steal the data and decrypt it elsewhere.
  \end{solution}

  \part[3] Give a brief analysis of its advantages and disadvantages.
  \begin{solution}
    The data is secure.
    However, it might be too secure: what happens if the hardware breaks, e.g.\ 
    you need to buy a new computer?
  \end{solution}
\end{parts}
