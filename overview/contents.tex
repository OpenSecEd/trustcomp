\mode*

% Since this a solution template for a generic talk, very little can
% be said about how it should be structured. However, the talk length
% of between 15min and 45min and the theme suggest that you stick to
% the following rules:  

% - Exactly two or three sections (other than the summary).
% - At *most* three subsections per section.
% - Talk about 30s to 2min per frame. So there should be between about
%   15 and 30 frames, all told.


% XXX extend part on trusted computing
\section{Trusted computing}

\subsection{Desired properties}

\begin{frame}
  \begin{block}{The idea}
    \begin{itemize}
      \item What if a program running in a system could ascertain the integrity 
        of the system?

      \item E.g.\ that we run a particular OS, that the OS is unmodified, that 
        the program itself is unmodified.

    \end{itemize}
  \end{block}
\end{frame}

\begin{frame}
  \begin{block}{Remote attestation}
    \begin{itemize}
      \item We add a tamper-resistant hardware chip.
      \item This chip can query the rest of the hardware.
      \item It can then create a digitally signed summary of the hardware and 
        attest that it is correct.

        \pause{}

      \item We can even attest the running software.
    \end{itemize}
  \end{block}
\end{frame}

\begin{frame}
  \begin{block}{Sealed Storage}
    \begin{itemize}
      \item Protects private data by binding it to the platform.
        \begin{itemize}
          \item Use the hardware chip for encryption.
          \item The chip includes the configuration as part of the key.
          \item Only the chip has the key.
        \end{itemize}
    \end{itemize}
  \end{block}
\end{frame}

\begin{frame}
  \begin{example}
    \begin{itemize}
      \item Encrypt your own data, no one can steal it and decrypt it 
        elsewhere.
      \item If you change your hardware too much, then neither can you.
    \end{itemize}
  \end{example}

  \pause{}

  \begin{example}
    \begin{itemize}
      \item Encrypt media content with certain requirements.
      \item The hardware will only decrypt it if you run an unmodified version 
        of a DRM-enforcing player.
    \end{itemize}
  \end{example}
\end{frame}

\subsection{Trusted Platform Module}

\begin{frame}
  \begin{block}{Trusted Platform Module, TPM}
    \begin{itemize}
      \item The Trusted Platform Module (TPM) is an industry standard.
      \item It is maintained by the Trusted Computing Group (TCG).
      \item Most computers has such a chip.
    \end{itemize}
  \end{block}
\end{frame}

\begin{frame}
  \begin{block}{TPM functionality}
    \begin{itemize}
      \item Random number generator
      \item Generation of crypto keys
      \item Remote attestation
      \item Binding: encrypts data with the TPM binding key.
      \item Sealing: similar to binding, but specifies TPM state for 
        decryption.
    \end{itemize}
  \end{block}
\end{frame}

\subsection{Intel SGX secure enclaves}

\begin{frame}
  \begin{block}{Idea behind secure enclaves}
    \begin{itemize}
      \item Consider operating system, hypervisor hostile.
      \item An enclave is a secure computing environment in this environment.
    \end{itemize}
  \end{block}
\end{frame}

\begin{frame}
  \begin{block}{Highlights of Intel SGX}
    \begin{itemize}
      \item The enclave memory (data and code) is encrypted.
      \item Decrypted on-the-fly \emph{inside and only inside the processor}.
      \item Only decrypted for code inside the enclave.
      \item Remote attestation for enclaves.
      \item Sealed storage.
    \end{itemize}
  \end{block}
\end{frame}

\begin{frame}
  \begin{example}[Uses]
    \begin{itemize}
      \item Remote computation: encrypt code, data for enclave, send to remote.
      \item Secure web browsing: lock things into an enclave.
      \item Digital rights management
      \item Conceal proprietary algorithms and encryption keys.
    \end{itemize}
  \end{example}
\end{frame}

\begin{frame}
  \begin{example}[Signal contact discovery~\cite{SignalSGXContactDiscovery}]
    \begin{itemize}
      \item User phone runs an enclave on Signal's server.
      \item Sends contacts encrypted for attested enclave.
      \item Enclave looks up if contacts are registered users.
      \item Returns encrypted results.
    \end{itemize}
  \end{example}

  \pause

  \begin{remark}
    \begin{itemize}
      \item Must be careful how to do it!
      \item See \cite{SignalSGXContactDiscovery} for details:
        \begin{center}
          \href{https://signal.org/blog/private-contact-discovery/}
            {signal.org/blog/private-contact-discovery}.
        \end{center}
    \end{itemize}
  \end{remark}
\end{frame}

\begin{frame}
  \begin{example}[Implementations]
    \begin{itemize}
      \item Intel SGX: from 6th gen Intel Core processors.
      \item QEMU supports Intel SGX.
      \item Georgia Tech's OpenSGX simulator.
    \end{itemize}
  \end{example}
\end{frame}

\begin{frame}
  \begin{remark}
    \begin{itemize}
      \item There are attacks against Intel SGX!
    \end{itemize}
  \end{remark}
\end{frame}

\subsection{Root of trust}

\begin{frame}
  \begin{remark}
    \begin{itemize}
      \item We must root our trust somewhere.
    \end{itemize}
  \end{remark}

  \pause

  \begin{block}{Two approaches}
    \begin{enumerate}
      \item Trust hardware manufacturer, its firmware is read-only.
      \item We test it.
    \end{enumerate}
  \end{block}
\end{frame}

\begin{frame}
  \begin{example}[We trust the manufacturer]
    \begin{itemize}
      \item We embed a digital certificate in the hardware.
      \item Hardware can sign all output.
      \item Then we can be sure we're talking to the correct hardware.
    \end{itemize}
  \end{example}
\end{frame}

\begin{frame}
  \begin{example}[We test the hardware]
    \begin{itemize}
      \item We must test the hardware (software onboard) in a particular 
        way~\cite{EstablishRootOfTrustUnconditionally}.

      \item System we test has memory \(M\) and registers \(R\).
    \end{itemize}
    \begin{enumerate}
      \item Verifier asks system to initialize \(M, R\).

      \item Verifier asks system to compute \(C_n = C_{m,t}(M, R, n)\) for some 
        nonce \(n\), takes \(m\) words and \(t\) time.
    \end{enumerate}
  \end{example}
\end{frame}

\begin{frame}
  \begin{remark}
    \begin{itemize}
      \item If \(C_{m,t}\) is
        \begin{itemize}
          \item space-time optimal\footnote{%
              Lower bound equals upper bound for the algorithm.
            } and
          \item second preimage free,
        \end{itemize}
      \item then this cannot be done in fewer than \(m\) words and less than 
        \(t\) time.
      \item All chips in the system must be tested simultaneously for the same 
        amount of time, so they can't help each other.
    \end{itemize}
  \end{remark}
\end{frame}

\begin{frame}
  \begin{remark}
    \begin{itemize}
      \item The details are in \fullcite{EstablishRootOfTrustUnconditionally}.
      \item There is a video presentation of the paper here:
        \begin{center}
          \url{https://youtu.be/CLQosBDiG3Q}.
        \end{center}
    \end{itemize}
  \end{remark}
\end{frame}
