\mode*

% Since this a solution template for a generic talk, very little can
% be said about how it should be structured. However, the talk length
% of between 15min and 45min and the theme suggest that you stick to
% the following rules:  

% - Exactly two or three sections (other than the summary).
% - At *most* three subsections per section.
% - Talk about 30s to 2min per frame. So there should be between about
%   15 and 30 frames, all told.


% XXX extend part on trusted computing
\section{Trusted Computing}

\subsection{Desired Properties}

\begin{frame}
  \begin{block}{The idea}
    \begin{itemize}
      \item What if a program running in a system could ascertain the integrity 
        of the system?

      \item E.g.\ that we run a particular OS, that the OS is unmodified, that 
        the program itself is unmodified.

    \end{itemize}
  \end{block}
\end{frame}

\begin{frame}
  \begin{block}{Remote attestation}
    \begin{itemize}
      \item We add a tamper-resistant hardware chip.
      \item This chip can query the rest of the hardware.
      \item It can then create a digitally signed summary of the hardware and 
        attest that it is correct.

        \pause{}

      \item We can even attest the running software.
    \end{itemize}
  \end{block}
\end{frame}

\begin{frame}
  \begin{alertblock}{Dangers}
    \begin{itemize}
      \item This could be used to lock the user out of the hardware.
        \begin{itemize}
          \item Run authentic Windows or don't use the hardware at all!
          \item Linux?!
            Anything you create yourself?!
            If you're not a multimillion dollar company, who cares?
        \end{itemize}
    \end{itemize}
  \end{alertblock}
\end{frame}

\begin{frame}
  \begin{block}{Sealed Storage}
    \begin{itemize}
      \item Protects private data by binding it to the platform.
        \begin{itemize}
          \item Use the hardware chip for encryption.
          \item The chip includes the configuration as part of the key.
          \item Only the chip has the key.
        \end{itemize}
    \end{itemize}
  \end{block}
\end{frame}

\begin{frame}
  \begin{example}
    \begin{itemize}
      \item Encrypt your own data, no one can steal it and decrypt it 
        elsewhere.
      \item If you change your hardware too much, then neither can you.
    \end{itemize}
  \end{example}

  \pause{}

  \begin{example}
    \begin{itemize}
      \item Encrypt media content with certain requirements.
      \item The hardware will only decrypt it if you run an unmodified version 
        of a DRM-enforcing player.
    \end{itemize}
  \end{example}
\end{frame}

\subsection{Trusted Platform Module}

\begin{frame}
  \begin{block}{Trusted Platform Module, TPM}
    \begin{itemize}
      \item The Trusted Platform Module (TPM) is an industry standard.
      \item It is maintained by the Trusted Computing Group (TCG).
      \item Most computers has such a chip.
    \end{itemize}
  \end{block}
\end{frame}

\begin{frame}
  \begin{block}{TPM functionality}
    \begin{itemize}
      \item Random number generator
      \item Generation of crypto keys
      \item Remote attestation
      \item Binding: encrypts data with the TPM binding key.
      \item Sealing: similar to binding, but specifies TPM state for 
        decryption.
    \end{itemize}
  \end{block}
\end{frame}

