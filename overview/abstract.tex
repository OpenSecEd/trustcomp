One can only do so much with software.
The problem with software and general purpose processors is that the software 
can be modified and the processor will still execute it.
Here we will explore how to ensure the integrity of the computer system before 
use.
As an example, Alice has a laptop while travelling, how can she be sure no 
foreign intelligence agency inserted a modified version of the operating system 
during the customs inspection?
Or, what about when she left the laptop in the hotel room while having 
breakfast, perhaps the hotel aide replaced the bootloader to break Alice's 
full-disk encryption?
Another aspect of this is to protect parts of the system from Alice herself, 
this is what Digital Rights Management is all about.
A content owner who only allows using his or her material in a certain way must 
have some means of ensuring this is enforced.
These needs boils down to trusted computing.

We treat the material in Chapters 16, 18 and 22 in 
\citetitle{Anderson2008sea}~\cite{Anderson2008sea}.
