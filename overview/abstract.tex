One can only do so much with software.
The problem with software and general purpose processors is that the software 
can be modified and the processor will still execute it.
Some examples: Alice left the laptop in the hotel room while having breakfast, 
perhaps the hotel aide replaced the bootloader to break Alice's full-disk 
encryption?
Or, how can Alice even trust the computer when it is brand new?
Another aspect of this is to protect parts of the system from Alice herself, 
e.g.\ this is what \ac{DRM} is all about.
We also have the compartmentalization of apps in a smartphone.
If Alice accidentally installs a malicious app, it shouldn't be able to 
compromize the banking app.
Here we will explore how to ensure the integrity of the computer system.

More concretely, after this session you should be able to
\begin{itemize}
  \item \emph{understand} the problem of trusted computing, its approaches to 
    solutions, the underlying assumptions and its limitations.
  \item \emph{analyse} different approaches to trusted computing and their 
    limitations and \emph{apply} them in a solution to a given problem.
\end{itemize}

We treat the material in Chapters 16, 17, 18, 22 and 23 in 
\citetitle{Anderson2008sea}~\cite{Anderson2008sea}.
The papers~\cite{%
  VoltageKeyExtraction,AcousticKeyExtraction,ElectromagneticKeyExtraction%
} is an interesting example which illustrates just how difficult this can be.
The authors extract encryption keys using acoustic side-channels, i.e.\ they 
analyse the sound emitted by the electrical circuitry to find the computations 
done and hence derive the bits of the key used.
