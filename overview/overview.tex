%\documentclass[handout]{beamer}
\documentclass{beamer}
\usepackage[utf8]{inputenc}
\usepackage[T1]{fontenc}
\usepackage[ibycus,swedish,english]{babel}
\usepackage{url}
\usepackage{graphicx}
\usepackage{color}
\usepackage{subfig}
\usepackage{multicol}
\usepackage{amssymb,amsmath,amsthm}
\usepackage{booktabs}
\usepackage[squaren,binary]{SIunits}
\usepackage{verbatim}
\usepackage{listings}
\usepackage{csquotes}

\setbeamertemplate{bibliography item}[text]
\usepackage[natbib,style=alphabetic,maxbibnames=99]{biblatex}
\addbibresource{overview.bib}

\newenvironment{axiom}[1]{\begin{block}{Axiom (#1)}}{\end{block}}

\DeclareMathOperator{\U}{\mathcal{U}}
\DeclareMathOperator{\T}{\mathcal{T}}
\DeclareMathOperator{\V}{\mathcal{V}}

\DeclareMathOperator{\N}{\mathbb{N}}
\DeclareMathOperator{\Z}{\mathbb{Z}}
\DeclareMathOperator{\R}{\mathbb{R}}

\let\stoch\mathbf\relax

\DeclareMathOperator{\lequiv}{\Longleftrightarrow}
\DeclareMathOperator{\xor}{\oplus}

\renewcommand{\qedsymbol}{Q.E.D.}

\DeclareMathOperator{\hmac}{HMAC}
\DeclareMathOperator{\concat}{||}

\DeclareMathOperator{\believes}{|\!\!\!\equiv}
\DeclareMathOperator{\said}{|\!\!\!\sim}
\DeclareMathOperator{\controls}{\Mapsto}
\DeclareMathOperator{\sees}{\lhd}
\newcommand{\fresh}[1]{\#(#1)}
\newcommand{\encrypt}[2]{\{#1\}_{#2}}
\newcommand{\share}[1]{\stackrel{#1}{\leftrightarrow}}
\newcommand{\pubkey}[1]{\stackrel{#1}{\mapsto}}

\mode<presentation>{%
  \usetheme{Frankfurt}
  \setbeamercovered{transparent}
  \usecolortheme{seagull}
}
\setbeamertemplate{footline}{\insertframenumber}

\title{%
  Trusted Computing
}
\author{Daniel Bosk\footnote{%
  This work is licensed under the Creative Commons Attribution-ShareAlike 3.0 
  Unported license.
	To view a copy of this license, visit 
	\url{http://creativecommons.org/licenses/by-sa/3.0/}.
}}
\institute[MIUN ICS]{%
  Department of Information and Communication Systems,\\
  Mid Sweden University, Sundsvall.
}
\date{\today}

%\pgfdeclareimage[height=0.65cm]{university-logo}{MU_logotyp_int_CMYK.pdf}
%\logo{\pgfuseimage{university-logo}}

\AtBeginSection[]{%
  \begin{frame}<beamer>{Overview}
    \begin{multicols}{2}
      \tableofcontents[currentsection]
    \end{multicols}
  \end{frame}
}
%\AtBeginSubsection[]{%
%  \begin{frame}<beamer>{Overview}
%    \tiny
%    \tableofcontents[currentsubsection,sectionstyle=shaded]
%  \end{frame}
%}

\begin{document}

\begin{frame}
  \titlepage{}
\end{frame}

%\begin{frame}{Literature}
%  \input{literature.tex}
%\end{frame}
\begin{frame}{Overview}
  \begin{multicols}{2}
    \tableofcontents
  \end{multicols}
\end{frame}


% Since this a solution template for a generic talk, very little can
% be said about how it should be structured. However, the talk length
% of between 15min and 45min and the theme suggest that you stick to
% the following rules:  

% - Exactly two or three sections (other than the summary).
% - At *most* three subsections per section.
% - Talk about 30s to 2min per frame. So there should be between about
%   15 and 30 frames, all told.


\section{Digital Rights Management}

\subsection{What is DRM?}

\begin{frame}{\insertsubsectionhead}
  \begin{itemize}
    \item The main purpose of DRM is to prevent piracy.

    \item This can be applied to all sorts of material; from photos, to films, 
      to application programs, and all the way to operating systems.

    \item There are different approaches and purposes, e.g.\ to control piracy, 
      but also to control the selling of used products.

  \end{itemize}
\end{frame}

\subsection{Historical Approaches}

\begin{frame}{\insertsubsectionhead}
  \begin{itemize}
    \item In the dawn of computing software was given away for free by the 
      hardware (HW) vendors.

    \item This was one way to promote sales of HW, the users needed software to 
      use the HW\@.

    \item This changed, and in the 1960's software was a significant cost.

    \item Now HW vendors charged extra for their OSes and there were 
      third-party software vendors.
  \end{itemize}
\end{frame}

\begin{frame}{\insertsubsectionhead}
  \begin{itemize}
    \item In the 1970's software could be turned into general packages.

    \item I.e.\ software needed no longer be customised to the users' HW\@.

    \item Now problems with the ownership of code rose, what if one of your 
      programmers left for a competitor and their program soon got some of your 
      features.

    \item To determine if the programmer copied the source or reinvented it, 
      software birthmarks could be used -- i.e.\ analysing how the software is 
      coded.
  \end{itemize}
\end{frame}

\begin{frame}{\insertsubsectionhead}
  \begin{itemize}
    \item Then came the 1980's, with these general purpose computer systems 
      came attempts at copyright enforcement.

    \item Some approaches was to lock the software with an error message every 
      few months, e.g.\ ``Error \(X\):  Please call technical support'', where 
      \(X\) is a customer specific number.

    \item This worked for as long as users were technically unknowledgable and 
      it didn't cross the limit what was considered reliable.

    \item Other apporaches was for the software to look at the processor's 
      serial number.
  \end{itemize}
\end{frame}

\begin{frame}{\insertsubsectionhead}
  \begin{itemize}
    \item In summary, there was essentially three general approaches tried.

    \item First, to add uniqueness to the machine; e.g.\ a HW dongle.

    \item Second, to create uniqueness within it; e.g.\ install the software in 
      a way that prevented naïve copying (cf.\ Adobe Photoshop which modified 
      the boot loader and accidentally removes Grub).

    \item Generally people must be able to create a backup, but not copy those 
      backups for sharing (copy generation control).

    \item And third, to use whatever uniqueness there already was; e.g.\ 
      storing the characteristics of the computer, cards present, amount of 
      memory, etc.

    \item This approach needs to handle HW upgrades though.
  \end{itemize}
\end{frame}

\subsection{Modern Approaches}

\begin{frame}{\insertsubsectionhead}
  \begin{itemize}
    \item One of the more modern approaches is to have the software connect to 
      the vendor's servers to verify itself.

    \item This works as long as the software isn't needed offline.

    \item But even online it can be really annoying, cf.\ Ubisoft's Assassin's 
      Creed DRM which required a constant connection.

    \item Another is to leave some critical part to be done by the vendor's 
      servers.

    \item An example of this is Blizzard's Diablo 3 games, which lets the 
      server handle the entire game (map generation, NPCs, etc.).
  \end{itemize}
\end{frame}

\begin{frame}{\insertsubsectionhead}
  \begin{itemize}
    \item Yet other approaches is to encrypt vital parts, e.g.\ some code or 
      video.

    \item This can be used for both software and media, for which it is popular 
      (DVD, BlueRay, streaming services).

    \item However, this must be decrypted before use \dots
  \end{itemize}
\end{frame}


% XXX remove part on secure protocols
\section{Secure Protocols}

\subsection{What is a protocol?}

\begin{frame}{\insertsubsectionhead}
  \begin{itemize}
    \item Ett system består av en uppsättning principals.
    \item Ett protokoll är en uppsättning regler som styr hur dessa 
      kommunicerar.
  \end{itemize}
  \begin{example}[Tentamen MIUN]
    \begin{enumerate}
      \item Tentamensvakten öppnar salen och ger varje tentand ett nummer.
      \item Tentanden går in och sätter sig vid sin tilldelade plats.
      \item Efter att tentan börjat jämför tentamensvakten tentandens 
        legitimation och nummer.
      \item Vid inlämning av skrivning jämförs legitimationen och numret.
    \end{enumerate}
  \end{example}
\end{frame}

\begin{frame}{\insertsubsectionhead}
  \begin{itemize}
    \item Bör vara designade för att motstå attacker.
    \item Både oavsiktligt och avsiktligt brott mot protokollet.
  \end{itemize}
\end{frame}

\begin{frame}{\insertsubsectionhead}
  \begin{itemize}
    \item Konstrueras utifrån grundläggande antaganden.
      \begin{itemize}
        \item Exempelvis att kortägaren kan mata in PIN-koden direkt 
          i terminalen.
    \end{itemize}
    \item Analysera om hoten är rimliga.
    \item Analysera om protokollet hanterar dem.
  \end{itemize}
\end{frame}

\subsection{Formell notation}

\begin{frame}{\insertsubsectionhead}
  \begin{example}[Protokollbeskrivning]
    Två principals \(P, P^\prime\) ska kommunicera.
    \begin{enumerate}
      \item \(P\) skickar sitt namn till \(P^\prime\).
      \item \(P^\prime\) svarar med ett token \(t_P\) för vidare användning, 
        detta är krypterat med \(P\):s kryptonyckel \(k_P\).
    \end{enumerate}
  \end{example}
  \begin{example}[Formell beskrivning]
    Principals \(P, P^\prime\), token \(t_P\), \(P\):s kryptonyckel \(k_P\).
    \begin{align*}
      P\to P^\prime&\colon P \\
      P^\prime\to P&\colon \{t_P\}_{k_P}
    \end{align*}
  \end{example}
\end{frame}

\begin{frame}{\insertsubsectionhead}{Tentamen}
  \begin{example}[Autentisering MIUN]
    Låt \(T\) vara tentanden, \(V\) tentamensvakten, \(n_T\) det unika numret 
    för \(T\) och \(S\) skrivningen.
    Vidare låt \(k\) vara en kryptonyckel delad mellan legitimationsutfärdaren 
    och tentamensvakten (legitimation).
    \begin{align*}
      V\to T&\colon n_T \\
      T\to V&\colon \{T\}_k, n_T \\
      T\to V&\colon \{T\}_k, n_T, S
    \end{align*}
  \end{example}
\end{frame}

\subsection{Protokoll och attacker}

\begin{frame}{\insertsubsectionhead}{En bättre metod för fjärrlås}
  \begin{example}[Fjärrlås]
    Låt \(A, B\) vara principals, \(n\) nonce, \(k_A\) en nyckel unik för 
    \(A\).
    \begin{align*}
      A\to B\colon A, \encrypt{A, n}{k_A}
    \end{align*}
  \end{example}
  \begin{block}{Egenskaper}
    \begin{itemize}
      \item Nonce \(n\) för färskhet.
      \item Krypteringen för identifiering.
    \end{itemize}
  \end{block}
\end{frame}

\begin{frame}{\insertsubsectionhead}{Nyckelhantering}
  \begin{itemize}
    \item Måste hantera nycklarna \(k_i\) för alla enheter \(i\).
    \item \emph{Nyckeldiversifiering}: huvudnyckel \(k_M\) och generera \(k_i 
      = \encrypt{i}{k_M}\).
    \item Måste tänka efter:
      \begin{itemize}
        \item 128-bitar nyckel krypterar 16-bitar ID, mindre lämpligt för 
          diversifiering.
        \item Svagt chiffer ger också dåligt resultat.
        \item \(k_i = i\xor k_M\)?
      \end{itemize}
  \end{itemize}
\end{frame}

\begin{frame}{\insertsubsectionhead}{Kolla nonces}
  Kolla nonces långt tillbaka i tiden.
  \begin{itemize}
    \item Jämför med senaste nonce.
    \item Spela in två och spela upp dem varannan gång.
    \item Förbetalda elmätare, köp två laddningar och använd dem om vartannat.
  \end{itemize}
\end{frame}

\begin{frame}{\insertsubsectionhead}{Betjäntattacken}
  \begin{itemize}
    \item Hur genereras nonces?
    \item En person som har tillfällig åtkomst att generera tokens.
    \item Generera ett antal, använd dem senare.
    \item Exempelvis engångskoder för att logga in hos internetbanken.
    \item Attacken fungerar om nonces är (pseudo-) slumptal.
  \end{itemize}
\end{frame}

\begin{frame}{\insertsubsectionhead}{Kontra betjäntattacken}
  \begin{block}{Förbättring}
    \begin{itemize}
      \item Använd en räknare \(c\) som successivt ökas på.
      \item \(A\to B\colon A, \encrypt{A, c+1}{k_A}\), \(c = c+1\).
      \item Inget \(c^\prime \leq c\) accepteras.
    \end{itemize}
  \end{block}
  \begin{block}{Problem}
    \begin{itemize}
      \item Får inte ha jämförelsen \(c^\prime = c\), ger 
        synkroniseringsproblem.
      \item \(c\notin \Z_+\) utan \(c\in \Z_{2^x}\), för något \(x\in \N\): vid 
        något tillfälle blir då \(c+1 < c \pmod{2^x}\).
    \end{itemize}
  \end{block}
\end{frame}

\begin{frame}{\insertsubsectionhead}{Andra tillämpningar}
  \begin{itemize}
    \item Tillbehörskontroll: skrivare ändrar inställning från \unit{1200}{dpi} 
      till \unit{300}{dpi} om icke-originalbläckpatroner används.
    \item \enquote{Använd alltid godkända originaldelar}.
    \item Inte hålla angripare ute, utan hålla användare inne.
    \item Läs kapitel 7 \emph{Economics} i~\cite{Anderson2008sea} för vidare 
      diskussion.
  \end{itemize}
\end{frame}

\subsection{Challenge--response}

\begin{frame}{\insertsubsectionhead}
  \begin{block}{Grundläggande princip}
    Två principals \(A, B\) med gemensam nyckel \(k\) och nonce \(n\).
    \begin{align*}
      A\to B &\colon n \\
      B\to A &\colon \encrypt{B, n}{k}
    \end{align*}
  \end{block}
  \begin{block}{Problem}
    \begin{itemize}
      \item Dåliga (pseudo-) slumptalsgeneratorer, ger förutsägbara \(n\).
    \end{itemize}
  \end{block}
\end{frame}

\begin{frame}{\insertsubsectionhead}{Tvåfaktorautentisering}
  \begin{itemize}
    \item Ha användarnamn och lösenord.
    \item Komplettera med extern kod; exempelvis genererad av koddosa, SMS till 
      mobiltelefonen.
    \item Finns många varianter, kombinera två:
      \begin{itemize}
        \item Något du vet (lösenord),
        \item något du har (koddosa, mobiltelefon),
        \item något du är (biometrik).
      \end{itemize}
  \end{itemize}
\end{frame}

\begin{frame}{\insertsubsectionhead}{Tvåfaktorautentisering}
  \begin{block}{Protokoll (tvåfaktorautentisering med koddosa)}
    Låt \(A, B, D\) vara principals, \(D\) är koddosa, \(k\) är nyckel delad 
    mellan \(B, D\) och \(p\) är \(A\):s PIN-kod.
    \begin{align*}
      A\to B &\colon A \\
      B\to A &\colon n \\
      A\to D &\colon n, p \\
      D\to A &\colon \encrypt{n}{k} \\
      A\to B &\colon \encrypt{n}{k}
    \end{align*}
  \end{block}
\end{frame}

\begin{frame}{\insertsubsectionhead}{Tvåkanalsautentisering}
  \begin{block}{Protokoll (tvåkanalsautentisering med mobiltelefon)}
    Låt \(A, B, M\) vara principals, \(M\) är mobiltelefon och \(p\) är \(A\):s 
    lösenord.
    \begin{align*}
      A\to B &\colon A, p \\
      B\to M &\colon n \\
      M\to A &\colon n \\
      A\to B &\colon n
    \end{align*}
  \end{block}
\end{frame}

\subsection{Miljöbyte}

\begin{frame}{\insertsubsectionhead}
  \begin{itemize}
    \item Betalkortsystemet designades för en pålitlig miljö.
    \item Kraftigt reglerad miljö inbyggd i bankens fasad.
    \item Tillämpas i den mindre pålitliga miljön i samtliga affärer.
    \item Skimming.
  \end{itemize}
\end{frame}

\begin{frame}{\insertsubsectionhead}{Personen i mitten}
  \begin{itemize}
    \item \enquote{Det är enkelt att spela oavgjort mot en schackstormästare 
        i postschack: spela bara mot två stormästare samtidigt, en som vit och 
        en som svart, och skicka deras brev mellan varandra.} (John Convey)
    \item Problem med pålitliga användargränssnitt: hur vet du att inte 
      kortterminalen ljuger?
  \end{itemize}
\end{frame}

\subsection{Internetbanken och betalkort}

\begin{frame}{\insertsubsectionhead}{Olika former av bankdosor}
  \begin{block}{Swedbank}
    \begin{itemize}
      \item Individuell dosa, förkonfigurerad av banken.
      \item Kan generera engångskod.
      \item Kan hantera challenge--response.
    \end{itemize}
  \end{block}
  \begin{block}{Nordea}
    \begin{itemize}
      \item Oberoende smartkortläsare, använder individuellt betalkort.
      \item Kan generera engångskod.
      \item Kan hantera challenge--response.
    \end{itemize}
  \end{block}
\end{frame}

\begin{frame}{\insertsubsectionhead}{Problem som kan uppstå}
  \begin{block}{Problem}
    \begin{itemize}
      \item Om bankkort och dosa förvaras tillsammans kan PIN-koden utläsas 
        från de slitna knapparna på bankdosan.
      \item Om kortet används i en dålig terminal har angriparna allt som 
        behövs för att logga in till ditt bankkonto.
    \end{itemize}
  \end{block}
  \begin{block}{Förbättringar}
    \begin{itemize}
      \item Använd inte samma säkerhetsmekanism i flera sammanhang.
      \item Ha separata oberoende mekanismer.
      \item Ha ett pålitligt användargränssnitt.
    \end{itemize}
  \end{block}
\end{frame}

% XXX add slides about BankID
%\subsection{BankID}
%\begin{frame}{\insertsubsectionhead}
%\end{frame}
%\begin{frame}{\insertsubsectionhead}{Att lämna in deklarationen}
%\end{frame}



% XXX extend part on trusted computing
\section{Trusted Computing}

\subsection{Trusted Platform Module}

\begin{frame}{\insertsubsectionhead}
\end{frame}


% XXX extend part on information hiding
\section{Information Hiding}

\subsection{Watermarking}

\begin{frame}{\insertsubsectionhead}
  \begin{itemize}
    \item A different approach has to be taken for non-executable content, 
      since this material cannot check itself.

    \item The approach here is watermarking using steganographic methods.

    \item However, these are also quite easily thwarted.
  \end{itemize}
\end{frame}



%%%%%%%%%%%%%%%%%%%%%%

\begin{frame}{Referenser}
  \small
  \printbibliography{}
\end{frame}

\end{document}
