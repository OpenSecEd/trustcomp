\mode*

% Since this a solution template for a generic talk, very little can
% be said about how it should be structured. However, the talk length
% of between 15min and 45min and the theme suggest that you stick to
% the following rules:  

% - Exactly two or three sections (other than the summary).
% - At *most* three subsections per section.
% - Talk about 30s to 2min per frame. So there should be between about
%   15 and 30 frames, all told.


\section{Digital Rights Management}

\subsection{What is DRM?}

\begin{frame}
  \begin{itemize}
    \item The main purpose of DRM is to prevent piracy.

    \item This can be applied to all sorts of material; from photos, to films, 
      to application programs, and all the way to operating systems.

    \item There are different approaches and purposes, e.g.\ to control piracy, 
      but also to control the selling of used products.

  \end{itemize}
\end{frame}

\subsection{Historical Approaches}

\begin{frame}
  \begin{itemize}
    \item In the dawn of computing software was given away for free by the 
      hardware (HW) vendors.

    \item This was one way to promote sales of HW, the users needed software to 
      use the HW\@.

    \item This changed, and in the 1960's software was a significant cost.

    \item Now HW vendors charged extra for their OSes and there were 
      third-party software vendors.
  \end{itemize}
\end{frame}

\begin{frame}
  \begin{itemize}
    \item In the 1970's software could be turned into general packages.

    \item I.e.\ software needed no longer be customised to the users' HW\@.

    \item Now problems with the ownership of code rose, what if one of your 
      programmers left for a competitor and their program soon got some of your 
      features.

    \item To determine if the programmer copied the source or reinvented it, 
      software birthmarks could be used -- i.e.\ analysing how the software is 
      coded.
  \end{itemize}
\end{frame}

\begin{frame}
  \begin{itemize}
    \item Then came the 1980's, with these general purpose computer systems 
      came attempts at copyright enforcement.

    \item Some approaches was to lock the software with an error message every 
      few months, e.g.\ ``Error \(X\):  Please call technical support'', where 
      \(X\) is a customer specific number.

    \item This worked for as long as users were technically unknowledgable and 
      it didn't cross the limit what was considered reliable.

    \item Other apporaches was for the software to look at the processor's 
      serial number.
  \end{itemize}
\end{frame}

\begin{frame}
  \begin{itemize}
    \item In summary, there was essentially three general approaches tried.

    \item First, to add uniqueness to the machine; e.g.\ a HW dongle.

    \item Second, to create uniqueness within it; e.g.\ install the software in 
      a way that prevented naïve copying (cf.\ Adobe Photoshop which modified 
      the boot loader and accidentally removes Grub).

    \item Generally people must be able to create a backup, but not copy those 
      backups for sharing (copy generation control).

    \item And third, to use whatever uniqueness there already was; e.g.\ 
      storing the characteristics of the computer, cards present, amount of 
      memory, etc.

    \item This approach needs to handle HW upgrades though.
  \end{itemize}
\end{frame}

\subsection{Modern Approaches}

\begin{frame}
  \begin{itemize}
    \item One of the more modern approaches is to have the software connect to 
      the vendor's servers to verify itself.

    \item This works as long as the software isn't needed offline.

    \item But even online it can be really annoying, cf.\ Ubisoft's Assassin's 
      Creed DRM which required a constant connection.

    \item Another is to leave some critical part to be done by the vendor's 
      servers.

    \item An example of this is Blizzard's Diablo 3 games, which lets the 
      server handle the entire game (map generation, NPCs, etc.).
  \end{itemize}
\end{frame}

\begin{frame}
  \begin{itemize}
    \item However, the Blizzard approach might cause problems.
    \item For how long do you intend to support that product?
      \begin{itemize}
        \item If I buy something, then I expect to be able to use it for as 
          long as I like.
        \item If you stop supporting it, and I need the product, I should be 
          allowed to at least reverse engineer it and use that.
      \end{itemize}
  \end{itemize}
\end{frame}

\begin{frame}
  \begin{itemize}
    \item Yet other approaches is to encrypt vital parts, e.g.\ some code or 
      video.

    \item This can be used for both software and media, for which it is popular 
      (DVD, BlueRay, streaming services).

    \item However, this must be decrypted before use \dots

    \item But I can at least use the stuff for as long as I like (or have 
      functioning equipment).
  \end{itemize}
\end{frame}

% XXX extend part on information hiding
\subsection{Information hiding and watermarking}

\begin{frame}
  \begin{itemize}
    \item A different approach has to be taken for non-executable content, 
      since this material cannot check itself.

    \item The approach here is watermarking using steganographic 
      methods.

    \item However, these are also quite easily thwarted.
  \end{itemize}
\end{frame}

